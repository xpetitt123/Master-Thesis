%%%%%%%%%%%%%%%%%%%%%%%%%%%%%%%%%%%%%%%%%%%%%%%%%%%%%%%%%%%%%%%%%%%%%%%%%%%%%%%%%%%%%%%%%%%%%%%%%%%%%%%%%%%%%%%%
% Sablona Bc+Mgr+RNDr (CZ) pro PriF MU                                                                       %%%
% Autor: Petr Zemanek (zemanekp@math.muni.cz)                                                                %%%
% Licence: tento soubor je k dispozici bez jakychkoli omezeni                                                %%%
% Pripominky, dotazy, namety smerujte na diskusni forum: https://is.muni.cz/auth/discussion/sablona-prif/    %%%
% Typeset in LaTeX-2e                                                                                        %%%
% Verze: 2.2 (11. prosince 2019)                                                                             %%%
%%%%%%%%%%%%%%%%%%%%%%%%%%%%%%%%%%%%%%%%%%%%%%%%%%%%%%%%%%%%%%%%%%%%%%%%%%%%%%%%%%%%%%%%%%%%%%%%%%%%%%%%%%%%%%%%

% \documentclass[11pt,a4paper,oneside,final]{book} %% PRO JEDNOSTRANNY TISK
% \documentclass[11pt,a4paper,twoside,final]{book} %% PRO OBOUSTRANNY TISK
% \documentclass[12pt,a4paper,oneside,final]{book} %% PRO JEDNOSTRANNY TISK
\documentclass[12pt,a4paper,twoside,final]{book} %% PRO OBOUSTRANNY TISK

%%%%%%%%%%%%%%%%%%%%%%%%%%%%%%%%%%%%%%%%%%%%%%%%%%%%%%%%%%%%%%%%%%%%%%%%%%%%%%%%%%%%%%%%%%%%%%%%%%%%%%
%%%%%%%%%%%%%%%%%%%%%%%%%%%%%%%%%%%%%% ZAKLADNI NASTAVENI %%%%%%%%%%%%%%%%%%%%%%%%%%%%%%%%%%%%%%%%%%%%

%%%%%%%%%%%%
%% Zvolte kodovani dokumentu
%%%%%%%%%%%%

\usepackage[utf8]{inputenc} 
% \usepackage[cp1250]{inputenc} %% NASTAVENI PRO WINDOWS
% \usepackage[T1]{fontenc} %% puvodni evropske fonty->spatne umistene hacky nad c, d s hackem ma za sebou mezeru
\usepackage[IL2]{fontenc} %fonty vyladene pro cestinu/slovenstinu

%%%%%%%%%%%%
%% Nastavte si jazyk dokumentu 
%% lze pouzit volbu babel (pro pdflatex a latex) nebo bez babelu (pro pdfcslatex a cslatex
%%
%% pri volbe bez babelu bude zlobit v ukazkovych souborech prikaz ``\shorthandoff{-}'' a ``\shorthandoff{-}''
%% ty staci vymazat a je to bez problemu
%%
%% pri volbe ``\usepackage[english]{babel}'' nefunguji ceske uvozovky pomoci \uv{..} - staci nahradit ``..''
%%%%%%%%%%%%

%\usepackage[czech]{babel}
% \usepackage[slovak]{babel}
\usepackage[english]{babel}
% \usepackage{czech}
% \usepackage{slovak}

%%%%%%%%%%%%
%% Muzete si zkusit zmenit font dokumentu - pak ale musite v souboru zmenit nektera nastaveni (nebot je nastaveno
%% \overfullrule10pt je preteceni textu ihned vyznaceno cernym obdelnikem)
%% Veskere nastaveni je testovano pro pouzitou volbu
%%%%%%%%%%%%

% \usepackage{bookman}
% \usepackage{charter}
% \usepackage{fourier}
% \usepackage{mathpazo}
% \usepackage{newcent}
\usepackage{palatino}
% \usepackage{utopia}
%\usepackage{mathptmx}  %% volne dostupny font Adobe Times Roman
%  \usepackage[mtbold,mtplusscr,mtpluscal]{mathtime} % komercni matematicky font, dostupne pouze na UMS


%%%%%%%%%%%%%%%%%%%%%%%%%%%%%%%%%%%%%%%%%%%%%%%%%%%%%%%%%%%%%%%%%%%%%%%%%%%%%%%%%%%%%%%%%%%%%%%%%%%%
%%%%%%%%%%%%%%%%%%%%%%%%%%%%%%%% BALIKY POTREBNE PRO SABLONU %%%%%%%%%%%%%%%%%%%%%%%%%%%%%%%%%%%%%%%

\usepackage{longtable}
%%% balik ``longtable`` je potrebny pro sazeni kapitoly s pouzitym znacenim, jinak neni potreba
%%% balik ``lipsum`` je pouze pro sablonu, aby se generoval nahodny text - pro samotnou praci je mozne jej odstranit

%%%%%%%%%%%%%%%%%%%%%%%%%%%%%%%%%%%%%%%%%%%%%%%%%%%%%%%%%%%%%%%%%%%%%%%%%%%%%%%%%%%%%%%%%%%%%%%%%%%%
%%%%%%%%%%%%%%%%%%%%%%%%%%%%% BALIKY POTREBNE PRO MATEMATIKU %%%%%%%%%%%%%%%%%%%%%%%%%%%%%%%%%%%%%%%

\usepackage{amsmath,amssymb,amsthm, graphicx, subfig, enumitem, tikz, smartdiagram, url, hyperref, tikz-cd, minted}

%%%%%%%%%%%%%%%%%%%%%%%%%%%%%%%%%%%%%%%%%%%%%%%%%%%%%%%%%%%%%%%%%%%%%%%%%%%%%%%%%%%%%%%%%%%%%%%%%
%%%%%%%%%%%%%%%%%% NASTAVENI POVINNYCH CASTI DLE SMERNICE DEKANA %%%%%%%%%%%%%%%%%%%%%%%%%%%%%%%%

%%%%%%%%%%%%%%%%%%%%%%%%%%%%%%%%%%%%%%%%%%%%%%%%%%%%%%
%% NASTAVTE KONKRETNI UDAJE V CESTINE A ANGLICTINE  %%
%%%%%%%%%%%%%%%%%%%%%%%%%%%%%%%%%%%%%%%%%%%%%%%%%%%%%%

\usepackage[Mgr,Barevne]{sci.muni.thesis}
%% Mozne volby:
%% Bc - pro bakalarskou praci
%% Mgr - pro diplomovou praci
%% RNDr - pro rigorozni praci
%% Barevne - pro dokument s barevnymi odkazy
%% Tisk - pro dokument v cernobile barve

\NazevUstavu{Ústav matematiky a statistiky}{Department of mathematics and statistics}

\RokOdevzdaniPrace{2025}

\AkademickyRok{2024/2025}

\Autor{Tomáš Petit}{Bc. Tomáš Petit}

\NazevPrace{Topological data analysis}{Topological data analysis}{Topological data analysis}
%% Pokud nazev prace obsahuje nejake matematicke formule, tak je kvuli velikosti fontu pouzit prikaz \scalebox a nazev
%% zalamat rucne - toto plati pouze pro nazev prace na titulni list
%% {Rozbor řešení rovnice \scalebox{1.15}{${y''=f(x)\,x^{\scalebox{0.4}2}}$} s~počátečními podmínkami
%% \scalebox{1.15}{$y(-\infty)=0$} a~\scalebox{1.15}{$y'(\infty)=0$}}
%%
%% pro psani druhe polozky, tj. {Název práce} je nutne mit na pameti, ze matematicke symboly nelze do zalozek v PDF
%% vlozit - proto slouzi prikaz \texorpdfstring{toto se vysazi}{toto se vlozi do zalozek v PDF} 
%% nektere symboly vlozit lze, viz kapitolu 50 v dokumentaci
%% http://mirrors.ctan.org/macros/latex/contrib/hyperref/hyperref.pdf
%% viz take http://orgmode.org/worg/org-symbols.html
%%

\VedouciPraceSTituly{prof. RNDr. Jan Slovák, DrSc.} %% Neni potreba pro rigorozni prace

\StudijniProgram{Matematika}{Mathematics}

\StudijniObor{Matematika}{Mathematics}

\PocetStran{??\,$+$\,??}

\KlicovaSlova{Topologie; Algebraická Topologie; Homologie; Persistentní Homologie; Topologická analýza dat; TDA;
Topologické Metody}{Topology; Algebraic Topology; Homology; Persistent Homology; Topological data analysis; TDA; Topological Methods}

\Abstrakty%
{V této diplomové práci se věnujeme stručnému úvody do oblasti topologické analýzy dat (TDA) a použití výpočetní topologie a geometrie. Práce je psána pro čtenáře, který nemá silné zázemí v topologii či algebře. V prvních dvou kapitolách představíme základy teorie k pochopení používaných metod. V kapitole třetí a čtvrté prezentujeme použítí vysvětlených metod na příkladech v jazyce R a Python. Zdrojový kod je uveden v repozitáři práce.}%
{In this thesis, we introduce the reader to the fields of topological data analysis (TDA), computational topology and geometry. This work is intended for people without a strong background in topology or algebra. The first two chapters are dedicated to presenting the theory needed to interpret the results of the most used methods. Chapters 3 and 4 cover examples and practical case studies in the R and Python programming languages. The source code can be found in the repository of this work.}

\TextPodekovani%
{Na tomto místě bych chtěl své rodině a rodičům za jejich neustálou podporu v mém studiu. Taktéž bych chtěl poděkovat vedoucímu své práce za mnohé kávy a diskuze, které proběhli během těch dvou let.}

\TextProhlaseni%
{Prohlašuji, že jsem svoji diplomovou práci vypracoval samostatně pod vedením vedoucího práce
s~využitím informačních zdrojů, které jsou v práci citovány.\\[10mm]

Prohlašuji, že jsem svoji rigorózní práci vypracoval samostatně s~využitím informačních zdrojů, které jsou v práci
citovány.}

\DatumProhlaseni{5. měsíce 2025}

%%%%%%%%%%%%
%% Konkrétní příklad
% \NazevUstavu{Ústav matematiky a statistiky}{Department of Mathematics and Statistics}
% 
% \RokOdevzdaniPrace{2012}
% 
% \AkademickyRok{2011/12}
% 
% \Autor{Petr Zemánek}{Mgr. Petr Zemánek, Ph.D.}
% 
% \NazevPrace
% {Rozbor řešení rovnice \scalebox{1.15}{${y''=f(x)\,x^{\scalebox{0.4}2}}$} s~počátečními podmínkami
% \scalebox{1.15}{$y(-\infty)=0$} a~\scalebox{1.15}{$y'(\infty)=0$}}
% {Rozbor řešení rovnice \texorpdfstring{$y''=f(x)\,x^2$}{y''=f(x)x\texttwosuperior} s~počátečními podmínkami
% \texorpdfstring{$y(-\infty)=0$}{y(-∞)} a~\texorpdfstring{$y'(\infty)=0$}{y'(∞)=0}}
% {Analysis of solution of the equation $y''=f(x)\,x^2$ with the initial conditions $y(-\infty)=0$ and $y'(\infty)=0$}
% 
% {Lineární diferenciální rovnice {\it n}-tého řádu s~konstantními koeficienty a~jejich aplikace}{Linear
% Differential Equations of {\it n}-th Order with Constant Coefficients and Their Applications}
% 
% \VedouciPraceSTituly{Mgr. Petr Zemánek, Ph.D.}
% 
% \StudijniProgram{Matematika}{Mathematics}
% 
% \StudijniObor{Matematika}{Mathematics}
% 
% \PocetStran{xiii\,$+$\,14}  %%% tyto hodnoty odpovidaji teto sablone
%
% \KlicovaSlova{Klíčové slovo; Klíčové slovo; Klíčové slovo; Klíčové slovo; Klíčové slovo; Klíčové slovo; Klíčové
% slovo; Klíčové slovo}{Keyword; Keyword; Keyword; Keyword; Keyword; Keyword; Keyword; Keyword; Keyword}
% 
% \Abstrakty%
% {V této bakalářské/diplomové práci se věnujeme ...}%
% {In this thesis we study ...}
% 
% \TextPodekovani%
% {Na tomto místě bych chtěl(-a) poděkovat ...}
% 
% \TextProhlaseni{
% Prohlašuji, že jsem svoji bakalářskou práci vypracoval samostatně pod vedením vedoucího práce s~využitím 
% informačních zdrojů, které jsou v práci citovány.}
% 
% \DatumProhlaseni{27. ledna 2012}
%

%%%%%%%%%%%%%%%%%%%%%%%%%%%%%%%%%%%%%%%%%%%%%%%%%%%%%%%%%%%%%%%%%%%%%%%%%%%%%%%%%%%%%%%%%%%%%%%%%%%%%%
%%%%%%%%%%%%%%%%%%%%%%%%%%%%%%%%%%%% VLASTNI PRIKLAZY %%%%%%%%%%%%%%%%%%%%%%%%%%%%%%%%%%%%%%%%%%%%%%%%
%%%%%%%%%%%%%% ZDE SI MUZETE DEFINOVAT VLASTNI PRIKAZY PRO SNAZSI SAZENI CELEHO TEXTU %%%%%%%%%%%%%%%%

\newcommand{\Cbb}{\mathbb{C}}
\newcommand{\Rbb}{\mathbb{R}}
\newcommand{\Zbb}{\mathbb{Z}}
\newcommand{\Nbb}{\mathbb{N}}
\DeclareMathOperator{\tg}{tg}

%%%%%%%%%%%%%%%%%%%%%%%%%%%%%%%%%%%%%%%%%%%%%%%%%%%%%%%%%%%%%%%%%%%%%%%%%%%%%%%%%%%%%%%%%%%%%%%%%%%%%%
%%%%%%%%%%%%%%%%%%%%%%%% VYTVOR POMOCNY SOUBOR PRO REJSTRIK %%%%%%%%%%%%%%%%%%%%%%%%%%%%%%%%%%%%%%%%%%

\makeindex

%%%%%%%%%%%%%%%%%%%%%%%%%%%%%%%%%%%%%%%%%%%%%%%%%%%%%%%%%%%%%%%%%%%%%%%%%%%%%%%%%%%%%%%%%%%%%%%%%%%%%%
%%%%%%%%%%%%%%%%%%%%%%% ZAPNUTI OPAKOVANI MATEMATICKYCH SYMBOLU %%%%%%%%%%%%%%%%%%%%%%%%%%%%%%%%%%%%%%

% \input{opakuj.sty}

%%%%%%%%%%%%%%%%%%%%%%%%%%%%%%%%%%%%%%%%%%%%%%%%%%%%%%%%%%%%%%%%%%%%%%%%%%%%%%%%%%%%%%%%%%%%%%%%%%%%%%
%%%%%%%%%%%%%%%%%%%%%%%%%%%%%%%%%% ZACATEK DOKUMETU %%%%%%%%%%%%%%%%%%%%%%%%%%%%%%%%%%%%%%%%%%%%%%%%%%
%%%%%%%%%%%%%%%%%%%%%%%%%%%%%%%%%%%%%%%%%%%%%%%%%%%%%%%%%%%%%%%%%%%%%%%%%%%%%%%%%%%%%%%%%%%%%%%%%%%%%%

\begin{document}

%%%%%%%%%%%%%%%%%%%%%%%%%%%%%%%%%%%%%%%%%%%%%%%%%%%%%%%%%%%%%%%%%%%%%%%%%%%
%%%%%%%%%%%% POVINNE CASTI PRO Bc. A Mgr. PRACE V CESTINE %%%%%%%%%%%%%%%%%

\VytvorPovinneStrany

%%%%%%%%%%%%%%%%%%%%%%%%%%%%%%%%%%%%%%%%%%%%%%%%%%%%%%%%%%%%%%%%%%%%%%%%%%%%%%%%%
%%%%%%%%%%%% POVINNE CASTI PRO Bc. a Mgr. PRACE PSANE SLOVENSKY %%%%%%%%%%%%%%%%%
% 
% \NazevPraceSLOVENSKY{...}
% 
% \NazevUstavuSLOVENSKY{...}
% 
% \StudijniProgramSLOVENSKY{...}
% 
% \StudijniOborSLOVENSKY{***}
% 
% \KlicovaSlovaSLOVENSKY{bbb}
% 
% \AbstraktSLOVENSKY{...}
% 
% \VytvorPovinneStranySLOVENSKY
%
%%%%%%%%%%%%%%%%%%%%%%%%%%%%%%%%%%%%%%%%%%%%%%%%%%%%%%%%%%%%%%%%%%%%%%%%%%%%%%%%
%%%%%%%%%%%% POVINNE CASTI PRO RIGOROZNI PRACE V CESTINE %%%%%%%%%%%%%%%%%%%%%%%

% \VytvorPovinneStranyRigorozniPrace

%%%%%%%%%%%%%%%%%%%%%%%%%%%%%%%%%%%%%%%%%%%%%%%%%%%%%%%%%%%%%%%%%%%%%%%%%%%%%%%%%%%%%%
%%%%%%%%%%%% POVINNE CASTI PRO RIGOROZNI PRACE PSANE SLOVENSKY %%%%%%%%%%%%%%%%%%%%%%%

% \VytvorPovinneStranyRigorozniPraceSVK

%%%%%%%%%%%%%%%%%%%%%%%%%%%%%%%%%%%%%%%%%%%%%%%%%%%%%%%%%%%%%%%%%%%%%%%%%%%%%%%%%%%%%%%%%%%%%%%%%%%
%%%%%%%%%%%%%%%%%%%%%%%%%%%%%%%%%% ABSTRAKT + ABSTRACT %%%%%%%%%%%%%%%%%%%%%%%%%%%%%%%%%%%%%%%%%%%%

\AbstraktyNaJedneStrane

%%%%%%%%%%%%%%%%%%%%%% POKUD SE ABSTRAKTY NEVEJDOU NA JEDNU STRANU, POUZIJTE TOTO NASTAVENI %%%%%%%

% \AbstraktyNaDvouStranach

%%%%%%%%%%%%%%%%%%%%%%%%%%%%%%%%%%%%%%%%%%%%%%%%%%%%%%%%%%%%%%%%%%%%%%%%%%%%%%%%%%%%%%%%%%%%%%%%%%%
%%%%%%%%%%%%%%%%%%%%%%%%%%%%%% ABSTRAKT + ABSTRAKT + ABSTRACT %%%%%%%%%%%%%%%%%%%%%%%%%%%%%%%%%%%%%

% \AbstraktyNaJedneStraneSLOVENSKY

%%%%%%%%%%%%%%%%%%%%%% POKUD SE ABSTRAKTY NEVEJDOU NA JEDNU STRANU, POUZIJTE TOTO NASTAVENI %%%%%%%

% \AbstraktyNaViceStranachSLOVENSKY


%%%%%%%%%%%%%%%%%%%%%%%%%%%%%%%%%%%%%%%%%%%%%%%%%%%%%%%%%%%%%%%%%%%%%%%%%%%%%%%%%%%%%%%%%%%%%%%%%%%
%%%%%%%%%%%%%%%%%%%%%%%%%%%%%%%%%%%%%% SEM VLOZIT ZADANI %%%%%%%%%%%%%%%%%%%%%%%%%%%%%%%%%%%%%%%%%%
%%% OSKENUJTE JEJ A VE FORMATU PDF VLOZTE DO STEJNE SLOZKY (SOUBOR POJMENUJTE NAPR. zadani.pdf) %%%
%%%%%% PAK ''ZAPROCENTUJTE'' RADEK S \SemVlozitZadani A ODPROCENTUJTE RADEK S \VlozZadani %%%%%%%%%
%%%%%%%%%%%%%%%%%%%%%%%%%%%%%%%%%%%%%%%%%%%%%%%%%%%%%%%%%%%%%%%%%%%%%%%%%%%%%%%%%%%%%%%%%%%%%%%%%%%

%\SemVlozitZadani

% \VlozZadani{NazevSouboruSeZadanim}
\VlozZadani{./is_zadání.pdf}

%%%%%%%%%%%%%%%%%%%%%%%%%%%%%%%%%%%%%%%%%%%%%%%%%%
%%%%%%%%%% PODEKOVANI + PROHLASENI %%%%%%%%%%%%%%%

\PodekovaniAProhlaseni

%%%%%%%%%%%%%%%%%%%%%%%%%%%%%%%%%%%%%%%%%%%%%%%%%%%%%%
%%%%%%%%%% PODEKOVANI + PROHLASENI SVK %%%%%%%%%%%%%%%

% \PodekovaniAProhlaseniSLOVENSKY

%%%%%%%%%%%%%%%%%%%%%%%%%%%%%%%%%%%%%%%%%%%%%%%%%%
%%%%%%%%%%%%% POUZE PROHLASENI %%%%%%%%%%%%%%%%%%%
%
% \ProhlaseniBezPodekovani
%
%%%%%%%%%%%%%%%%%%%%%%%%%%%%%%%%%%%%%%%%%%%%%%%%%%
%%%%%%%%%%%%%%%%%%% OBSAH %%%%%%%%%%%%%%%%%%%%%%%%

\pdfbookmark{Obsah}{Obsah}
\VytvorObsah
\cleardoublepage

%%%%%%%%%%%%%%%%%%%%%%%%%%%%%%%%%%%%%%%%%%%%%%%%%%%%%%%%%%%%%%%%%%%%%%%%%%%%%%%%%%%%%%%%%%%%%%%%%%%
%%%%%%%%%%%%%%%%%%%%%%%%%%%%%%%%%%% TEXT PRACE %%%%%%%%%%%%%%%%%%%%%%%%%%%%%%%%%%%%%%%%%%%%%%%%%%%%

%\HlavickaZnaceni
%\vloz{text_prace/01_Znaceni}
%\cleardoublepage

\HlavickaUvod
\setcounter{page}{1}
\pagenumbering{arabic}
\vloz{text_prace/02_Uvod}
\cleardoublepage

\renewcommand{\chaptermark}[1]{\markboth{\thechapter. #1}{}}
\renewcommand{\sectionmark}[1]{\markright{\thesection. #1}{}}
\HlavickaKapitoly
\vloz{text_prace/03_Kapitola_01}
\cleardoublepage

\vloz{text_prace/04_Kapitola_02}
\cleardoublepage

\vloz{text_prace/08_Kapitola_03}
\cleardoublepage

\vloz{text_prace/09_Kapitola_04}
\cleardoublepage

\HlavickaZaver
\vloz{text_prace/05_Zaver}
\cleardoublepage

%\HlavickaPriloha
%\vloz{text_prace/06_Priloha}
%\cleardoublepage

\renewcommand{\bibname}{Bibliography and sources}
\HlavickaLiteratura
\vloz{text_prace/07_Literatura}
\cleardoublepage

\renewcommand{\indexname}{Rejstřík}
\HlavickaRejstrik
\VytvorRejstrik  
%%
%% pro vytvoreni rejstriku se spravny ceskym razenim pouzijte 
%% csindex -d -h -k -z il2 nazev_souboru.idx
%%
%% nebo 
%% texindy -I latex -L czech -M lang/czech/utf8 nazev_souboru.idx
%%
%% na Ustavu matematiky a statistiky zadejte
%% /opt/texlive/2010/bin/i386-linux/texindy -I latex -L czech -M lang/czech/utf8 nazev_souboru.idx

%%%%%%%%%%%%%%%%%%%%%%%%%%%%%%%%%%%%%%%%%%%%%%%%
%%%%%%%%%%% PRAZDNA STRANA NA ZAVER %%%%%%%%%%%%
%%%%%%%%%%%%%%%%%%%%%%%%%%%%%%%%%%%%%%%%%%%%%%%%

\newpage
\thispagestyle{empty}
\fancyhf{}
\newpage
\mbox{}

\end{document}
