\chapter{TDA pipeline}
\graphicspath{ {/home/tomasp/Dokumenty/Master_Thesis/figures/} }

The goal of this chapter is to establish and set up the pipeline for extracting the algebraic invariants of our data. Usually, we can only work with sampled and discrete data coming from some set of measurements. As such we can't directly use methods of algebraic topology since we won't usually be working with discrete topological spaces and to properly use these methods, we would need an uncountable amount of data; something that isn't feasible from a computational point of view.
\par
This forces us to develop new methods to somehow approximate and recover the topology of the ambient space given only a finite set of points. Secondly, we also need to consider the \textit{scale} of the data - some interesting properties may be more apparent only after we ``zoom'' in closely on them, some may not become apparent at all. All in all, we have to first construct the following pipeline:

\begin{center}
\smartdiagram[sequence diagram]{
  Discrete data, Simplicial complex, Algebraic invariants}
\end{center}

and repeat this step for all scales at once, effectively measuring the evolution of the algebraic invariants through the changes in the feature scale.

\section{Associated Čech and VR complexes}
As mentioned above, most of the time in practice we work with a \textit{finite metric space}, that is, a metric space $(X, \partial_{X})$ with only finitely many points. Most often, this space is $\mathbb{R}^{n}$ with a collection of measurements $\{x_{1}, \ldots, x_{n}\}$ and the euclidean metric.
