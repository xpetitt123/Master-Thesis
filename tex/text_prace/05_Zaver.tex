\chapter*{Summary}
\addcontentsline{toc}{chapter}{Summary}

We spent the first two chapters building the theory behind the most difficult concept in Topological Data Analysis -- simplicial complexes and persistent homology. Once we established the concepts that make the theory usable in practice, that is, stability and robustness against noise, we began investigating its properties. In the practical part of the work, we showed examples of the three methods that are covered under the term Topological Data Analysis -- UMAP, Mapper, Persistent Homology and its equivalent representations. If the reader managed to get this far and understand the essence of it, they will be more than capable of picking up any article where those same methods are used.

To direct the reader in further and recent publications about TDA, we provide a short and incomplete list of articles that came out within the past two years that might be of interest: \cite{wei2025short}, \cite{ali2025leveraging}, \cite{jing2025topology}, \cite{jetomo2025filipino}, \cite{wiseman2025persistent}, \cite{arun2025topo}, \cite{mototaketopological}, \cite{pedersen2024active}, \cite{luchinsky2024tdavec}, \cite{hernandez2024topological}, \cite{cuerno2025topological}, \cite{arfi2024promises}, \cite{dos2025topological}, \cite{selicato2025topological}, \cite{boyd2024big}.

Additionally, a comprehensive review of some applications of the Mapper algorithm is in the article \cite{madukpe2025comprehensive}. If the reader wanted to join a community of researchers and scientists using methods from TDA and applied algebraic topology in general, they might want to visit \cite{AATRN}.

The entire repository of the thesis, along with all the figures, source code, and data sources, can be found on the following link: \url{https://github.com/xpetitt123/Master-Thesis}.
