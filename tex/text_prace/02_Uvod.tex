\chapter*{Introduction}
\addcontentsline{toc}{chapter}{Introduction}

Any thesis or research should answer one question before it is published, before it is even written -- Who is the intended audience? Knowing the target audience determines the structure and interpretation of the work itself. We can answer that question right now -- the ideal reader is someone working in statistics, data analysis, biology or any of the natural sciences who is interested in using a geometrical approach to a problem they might be looking at. People working in those areas usually do not have a rich background in algebraic topology or even topology itself. As such, we have made the decision to cut costs -- introduce the basic theory needed to interpret the results of computing the persistent homology of some object and how to work with those results further. This means that proofs are largely omitted and only referenced in textbooks or articles that the reader may wish to explore. The point of this work is to introduce someone from a non-topological field to Topological Data Analysis and teach them how to work with the methods, not drown them in technical details.
